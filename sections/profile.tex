\IfLanguageName{english}{
    \small
    {
        I graduated in TADS (Technology and Analysis in System Development) from Univel in 2008 and since then I have been working in the area of equipment, 
        server, and network maintenance, as well as user support. This role gave me access to Delphi system development. 
        In 2008, I took on the role of programmer and worked on ERP system development for poultry and swine refrigerators, as well as for the agribusiness area.\\

        From the beginning of my career, I have a affinity for the Linux operating system, which allowed me to learn more about datacenters. 
        In 2016, I had the opportunity to take on the DevOps role, where I led the implementation and automation of the pipeline, 
        as well as developed tools and scripts that are still used by the company. Since August 2019, I also lead the infrastructure team at Ecocentauro, 
        responsible for the physical restructuring of the new administrative headquarters, datacenter maintenance, internal network, and cloud.\\

        During the pandemic, I was responsible for the structure that allowed for the continuation of work from home for all employees, 
        without compromising the work scheme. Over the course of nearly 20 years in the profession, I have had the opportunity to work with various languages and technologies, 
        including Python, Golang, Dart, Containers, Bacula, VMWare ESXI, GitLab, AWS, and Kong.\\

        I enjoy the challenge my area provides, allowing me to improve the team's productive process, solve complex problems, 
        and anticipate possible situations that may affect the smooth operation of the company's work.
    }
}{
    \small
    {
        Eu me formei em TADS (Tecnologia e Análise em Desenvolvimento de Sistemas) pela Univel em 2008 e desde então tenho trabalhado na área de manutenção de equipamentos, 
        servidores e redes, bem como no suporte a usuários. Essa função me permitiu acesso ao desenvolvimento de sistemas em Delphi. 
        Em 2008, assumi a posição de programador e trabalhei no desenvolvimento de sistemas ERP para frigoríficos de aves e suínos, bem como para a área de agronegócios.\\

        Desde o início da minha carreira, tenho afinidade com o sistema operacional Linux, o que me permitiu aprender mais sobre datacenters. 
        Em 2016, tive a oportunidade de assumir a posição de DevOps, onde liderei a implantação e automação da pipeline, 
        além de desenvolver ferramentas e scripts que ainda são utilizados pela empresa. Desde agosto de 2019, também lidero a equipe de infraestrutura da Ecocentauro, 
        responsável pela reestruturação física da nova sede administrativa, manutenção do datacenter, rede interna e cloud.\\
        
        Durante a pandemia, fui responsável pela estruturação que permitiu a continuidade do trabalho em home office de todos os colaboradores, 
        sem prejudicar o esquema de trabalho. Ao longo de quase 20 anos na profissão, já tive a oportunidade de trabalhar com várias linguagens e tecnologias, 
        incluindo Python, Golang, Dart, Containers, Bacula, VMWare ESXI, GitLab, AWS e Kong.\\
        
        Eu gosto do desafio que a minha área me proporciona, de poder melhorar o processo produtivo da equipe, 
        solucionar problemas complexos e prever possíveis situações que possam afetar o bom andamento do trabalho da empresa.        
    }
}